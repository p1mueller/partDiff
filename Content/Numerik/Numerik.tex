\section{Numerik}
\subsection{Diskretisierung}
\subsubsection{1.Ableitung}

$$g'\approx \frac{g(x+\Delta x)-g(x)}{\Delta x} \qquad\qquad \text{oder} \qquad\qquad g'\approx \frac{g(x+\Delta x)-g(x-\Delta x)}{2\Delta x}\qquad \text{(Bessere Qualit�t)}$$
\subsubsection{2.Ableitung}

$g''\approx \frac{g(x+\Delta x)-2 g(x) + g(x- \Delta x)}{\Delta x^2}$

\subsection{FDM}
\textbf{TIPP:} Bei Anfangsbedingungen ungleich Null das Gleichungssystem selber von Hand herleiten, reduziert die Chance auf Fehler.
\subsubsection{Grundgleichung: $-u''(x)=f(x)$}
$ A^{(n)} \tilde{u}^{(n)} =f^{(n)}   $\\
$A^{(n)}= \frac{1}{\Delta x^2} tridiag {n-1} (-1,2,-1) = \frac{1}{\Delta x^2}
  \begin{bmatrix}
             2& -1 & 0 & \ldots \\
             -1& 2 & -1 & \ldots \\
              0& -1 & 2 & \ldots \\
              0& 0 & -1 & \ldots \\
             \ldots 
           \end{bmatrix} (eine (n-1)x(n-1)-Matrix)$\\ 
Randwert: $\tilde{u}(0)= a; \tilde{u}(n)=b $
$A^{(n)}=\tilde{u}^{(n)} =\begin{bmatrix}
             f(0) + \tilde{u}(0) \\
             f(1) \\
             \vdots  \\
             f(n) + \tilde{u}(n)
           \end{bmatrix} $\\
\subsubsection{Grundgleichung: $T''(x) + h (T_A -T(x)) = 0)$}
$-T'' + h T(x) = h T_A$\\
$A^{(n)}= \frac{1}{\Delta x^2} tridiag {n-1}
(2+h\Delta x^2 ) = \frac{1}{\Delta x^2}
  \begin{bmatrix}
             2+h\Delta x^2& -1 & 0 & \ldots \\
             -1& 2+h\Delta x^2 & -1 & \ldots \\
              0& -1 & 2+h\Delta x^2 & \ldots \\
              0& 0 & -1 & \ldots \\
             \ldots 
           \end{bmatrix} $\\       
           
           
           
\subsection{Konvergenz}
Ein Modell ist konvergent wenn bei $n\leftarrow\infty$ die Sch�tzung $\tilde{u}$ und $u$ �bereinstimmt.\\

$\boxed{||v||_{\Delta x}=\sqrt{\Delta x (v_1^2+v_2^2 + \ldots v_{n-1}^2)}= \sqrt{\Delta
x}||v||}$\\
Es konvertiert wenn: $\lim\limits_{n\to \infty
}||\tilde{u}^{(n)}-u^{(n)}||=0=\sqrt{\frac{1}{n}}||\tilde{u}^{(n)}-u^{(n)}=\sqrt{\frac{1}{n}}\sqrt{(\tilde{u}_1)-u_1)^2
+ \ldots + \tilde{u}_{n-1}-u_{n-1}}$\\

\subsection{Konsistenz}
Ein Modell ist nicht konsistent wenn das Modell durch Vereinfachung nicht mehr mit der Realit�t �bereinstimmt.

Globaler Konsistenzfehler: $\boxed{||r^{(n)}||_{1/n}\leq \frac 1{12}\max\limits_{\xi\in[0,1]}|f''(\xi)|\cdot \Delta x^2}$

\subsubsection{Residuum}
Exakt: $A^{(n)}\cdot \tilde{u}^{(n)}-f^{(n)}=0$\\
Residuum: $A^{(n)}\cdot (u^{(n)}-\tilde{u}^{(n)})=r^{(n)}$

Eine Approximationsverfahren ist Konsistent, wenn $\boxed{\lim\limits_{n\rightarrow \infty}||r^{(n)}||_{1/n}=0}$ gilt.\\

Konsistenz ist eine notwendige, aber nicht hinreichende Bedingung f�r die Konvergenz eines Verfahrens.

\subsubsection{Taylor}
$g(x)= \sum\limits_{k=0}^n\frac{1}{k!} g^{(n)}(x_0)(x-x_0)^k +
\frac{1}{(n+1)!}g^{(n+1)}(\xi)(x-x_0)^{n+1}$ = Taylor
Approximationspolynom  + Lagronesche Restglied ($\xi$ ist unabh�ngig von $x,
x_0$)



\subsubsection{Vorw�rt/R�ckw�rtsdifferenz}
$g'(x) - \frac{g(x+\Delta x) + g(x)}{\Delta x}= O(\Delta x) = 
\frac{g''(\xi_1)}{2}\Delta x^2 \Rightarrow 1.Ordnung$


\subsubsection{Zentraldifferenz}
$g'(x) - \frac{g(x+\Delta x) + g(x-\Delta x)}{2\Delta x}= O(\Delta x^2) = 
\frac{g''(\xi_1)}{2}\Delta x^2 \Rightarrow 2.Ordnung$ 



\subsubsection{2.Ableitung}
$g''(x) - \frac{g(x+\Delta x) -2 g(x)+ g(x-\Delta x)}{2\Delta x}= O(\Delta x^2)
= \frac{g''(\xi_1)}{2}\Delta x^2 \Rightarrow 1.Ordnung$

\subsection{Stabilit�t}
Die Stabilit�t einer Matrize kann �ber deren Norm $||A||_*$ bestimmt werden.\\

Es gilt: $||A||_*=\max\limits_{||X||_*=1}||A\cdot x||_*$\qquad$||A\cdot x||_*\leq||A||~||x||_*$\\

Ein Approximationsverfahren ist stabil wenn, wenn unabh�ngig von der konstante $C$ gilt:
$\boxed{||{A^{(n)}}^{-1}||\leq C}$\\



Die Bestimmung von $||A||$ ist im Allgemeinen nicht einfach, darum wird $||A||$ oft �ber den Umweg der Diagonalisierung von A bestimmt.

$y=A\cdot x\qquad\Rightarrow\qquad\tilde{y}=D\cdot\tilde{x}$\qquad mit\qquad $D=\begin{bmatrix}\lambda_1&&\\&\ddots&\\&&\lambda_n\end{bmatrix}$\\

Es gilt $TAT^T=D$, wobei T die Transformationsmatrix vom $x$-Koordinatensystem zum $\tilde{x}$-Koordinatensystem darstellt. $T$ ist orthogonal.
Die Diagonalelemente $\lambda_1,\ldots,\lambda_n$ werden auch Eigenwerte genannt.\\

Eine Approximationsformel ist stabil wenn: $\boxed{||A||=\max\limits_{k}|y_k|\leq C}$\\

Eigenwerte bestimmen: $\boxed{\det(A-\lambda I)=0}\qquad \Rightarrow \qquad \lambda_1,\ldots,\lambda_n$\\
Eigenvektoren bestimmen (f�r jedes $\lambda_i$): $A-\lambda_i I=0\qquad \Rightarrow \qquad v_1,\ldots,v_n$\\

\subsection{FDM f�r elliptisch PDGL (Poisson: $\Delta u = f$)}
%Dirichletsche Randbedingung ($u(x,y)= f(x,y) \forall (x,y)\epsilon \delta G)$\\

Gleichung:    $-\Delta u(x,y)= f(x,y) $
$\frac{g(x+\Delta x,y ) -2 g(x,y)+ g(x-\Delta x,y)}{2\Delta x} +
\frac{g(x,y+\Delta y) -2 g(x,y)+ g(x,y-\Delta y)}{2\Delta y}$\\

$h=\Delta x = \Delta y \Rightarrow \boxed{\frac 1 {h^2} (\tilde{u}_{j,k+1} +
\tilde{u}_{j+1,k} + \tilde{u}_{j,k-1} + \tilde{u}_{j-1,k} - 4 \tilde{u}_{j,k})
= f_{j, k}}$\\[0.4cm]


$B \tilde{u} = f \Rightarrow B= \begin{bmatrix}
             T& D & 0 & \ldots \\
             D& T & D & \ldots \\
              0& D & T & \ldots \\
              0& 0 & D & \ldots \\
             \ldots 
           \end{bmatrix}$
wobei $T=\begin{bmatrix}
             4& -1 & 0 & \ldots \\
             -1& 4 & -1 & \ldots \\
              0& -1 & 4 & \ldots \\
              0& 0 & -1 & \ldots \\
             \ldots 
           \end{bmatrix}$
und $D=\begin{bmatrix}
             -1& 0& 0 & \ldots \\
             0 & -1 &  & \ldots \\
              0& 0&-1 & \ldots \\
             \ldots 
           \end{bmatrix}$

           
\subsubsection{Irregul�re Gitter (f�r den Rand)}
\begin{minipage}{5cm}
	\includegraphics[width=5cm]{Content/Numerik/irregulaereGitter.png}

\end{minipage}
\begin{minipage}{14cm}

$\left(\frac{u(x+eh,y)-u(x,y)}{e(e+w)} +\frac{u(x+wh,y)-u(x,y)}{w(e+w)}+\frac{u(x+nh,y)-u(x,y)}{n(n+s)} + \frac{u(x+sh,y)-u(x,y)}{n(n+s)}\right)\frac{2}{h^2}=f(x,y)$\\

oder
\\

$\frac{u(P_E) - u(P)}{e(e+w)} + \frac{u(P_W) - u(P)}{w(e+w)} + \frac{u(P_N) - u(P)}{n(n+s)} + \frac{u(P_S) - u(P)}{s(n+s)} = \frac{h^2}{2} f(x,y)$
\end{minipage}
\subsubsection{Neumann Rand 
%$\partial_n u(x,y) \forall (x,y) \epsilon \partial_n G$
}
\begin{minipage}{8cm}
	\includegraphics[width=8cm]{Content/Numerik/NeumannRand.png}


\end{minipage}
\begin{minipage}{10cm}
In einem Randpunkt P liegen wie in der Abbildung ersichtlich, $P_N$ und $P_S$ auf dem Rand von Omega, $P_W$ liegt ausserhalb von Omega.\\
In P sei die Neumannsche Randbedingung: $\boxed{\partFrac{u}{n}(P)=g(P)}$\\
$u_x(P)=\frac{u(P_E)-u(P_W)}{2h}\quad\Rightarrow\quad u(P_W)=u(P_E)-2h\cdot u_x(P)$\\

$\boxed{\frac{2u(P_E) + u(P_N) +
u(P_S)- 4 u(P) - 2h\cdot u_x(P)}{h^2}}$\\


\textbf{Spiegelmethode}:\\
Wenn $u_x(x,y) = 0$, dann spricht man auch von der Spiegelmethode. Die Punkte $P_W$ und $P_E$ weisen dann die gleiche Wertigkeit auf ($P_W=P_E$).
\end{minipage}


\subsection{FDM f�r parabolische PDGL}
	W�rmeleitungsgleichung: $\boxed{u_t(x,t)=u_{xx}(x,t)}$\qquad $f(0)=f(1)=0$ \qquad$ \overset{\_}{\Omega}=[0,1]\times [0,\infty]$\\
	
	Randbedingungen: $u(x,0)=f(x) \qquad u(0,t)=u(1,t)=0\qquad x\in(0,1) \qquad t\in[0,\infty)$
\subsubsection{Explizites Verfahren (Richardson-Verfahren)}
$\boxed{\frac{\tilde{u}(x,t+\Delta t) - \tilde{u}(x,t)}{\Delta t} = 
\frac{\tilde{u}(x+\Delta x, t)-2\tilde{u}(x,y) + \tilde{u}( x - \Delta x, t )} {\Delta x^2}} \qquad \Delta x=\frac 1n\qquad \Delta t=\frac r{n^2} \qquad \boxed{r=\frac{\Delta
t}{\Delta x^2}}$\\

\textbf{Idee:} Aus den Positionen k wird k+1 berechnet: $\tilde{u}_{j,k+1} = r \tilde{u}_{j-1,k} + (1-2r)\tilde{u}_{j,k} + r \tilde{u}_{j+1,k}$\\


\begin{itemize}
\item Initialisierung, Randbedingung: $\tilde{u}_{j,0}=f(j/n)$ \qquad $\tilde{u}_{0,k}=\tilde{u}_{n,k}=0$
\item Approximationsmatrize: $C^{(n)}=tridiag_{n-1}(r,1-2r,r)=\begin{bmatrix}
1-2r& r		& 0		& 0 	&\cdots\\
r	& 1-2r  & r		& 0		&\cdots\\
0	& r		& 1-2r 	& r 	&\cdots\\
0	& 0		& r		& 1-2r 	&\cdots\\
\vdots&	\vdots&\vdots&\vdots&\ddots	
\end{bmatrix}$ 
\item Einen Schritt berechnen: $\tilde{u}^{(k+1)}=C^{(n)} \tilde{u}^{(k)}$
\item k-Schritte berechnen: $\tilde{u}^{(k)}=\big\{C^{(n)}\big\}^k \tilde{u}^{(0)}$
\end{itemize}

\textbf{Konvergenzverhalten:} \\

\begin{minipage}{6cm}
\includegraphics[width=6cm]{Content/Numerik/KonvExplizit.png}
\end{minipage}
\hfill
\begin{minipage}{12cm}
Verfahren ist stabil wenn: $||A^{-1}|| < 1 \qquad \Rightarrow\qquad r < \frac{1}{2}$\\

Dies macht es N�tig die Zeitschritte extrem klein zu w�hlen, darum ist das Verfahren auch nicht wirklich praxistauglich, weil sehr hohe Rechenkapazit�t n�tig sind.\\

Der Grund f�r das schlechte Konvergenzverhalten kann geometrisch visualisiert werden. In die Berechnung des Wertes im Knoten $P$, werden die Werte aller schwarz eingef�rbter Knoten eingehen. Von den Randwerten wird nur die 0-te Stufe ber�cksichtigt.
\end{minipage}






\subsubsection{Implizites Verfahren}
Im Unterschied zum expliziten Verfahren, das Werte vom vorherigen Zeitpunkt nutzt, wird hier das ein Gleichungssystem global gel�st.\\

$\boxed{\frac{\tilde{u}(x,t) - \tilde{u}(x,t -\Delta t)}{\Delta t} = 
\frac{\tilde{u}(x+\Delta x, t)-2\tilde{u}(x,y) + \tilde{u}( x - \Delta x, t )} {\Delta x^2}}$\\

$ \tilde{u}_{j,k} = r \tilde{u}_{j-1,k+1} + (1+2r)\tilde{u}_{j,k+1} + r \tilde{u}_{j+1,k+1}$

\textbf{Idee:} Die Ableitungen werden mittels R�ckw�rtsdifferenz berechnet\\


\begin{itemize}
\item Initialisierung, Randbedingung: $\tilde{u}_{j,0}=f(j/n)$ \qquad $\tilde{u}_{0,k}=\tilde{u}_{n,k}=0$
\item Approximationsmatrize: $E^{(n)}=tridiag_{n-1}(-r,1+2r,-r)=\begin{bmatrix}
1+2r& r		& 0		& 0 	&\cdots\\
r	& 1+2r  & r		& 0		&\cdots\\
0	& r		& 1+2r 	& r 	&\cdots\\
0	& 0		& r		& 1+2r 	&\cdots\\
\vdots&	\vdots&\vdots&\vdots&\ddots	
\end{bmatrix}$ 
\item Einen Schritt berechnen: $\tilde{u}^{(k+1)}=\left\{E^{(n)}\right\}^{-1} \tilde{u}^{(k)}$
\item k-Schritte berechnen: $\tilde{u}^{(k)}=\left(\left\{E^{(n)}\right\}^{-1}\right)^k \tilde{u}^{(0)}$
\end{itemize}

\textbf{Vorteil:} Das implizite Verfahren ist immer stabil, unabh�ngig von der Zeitaufl�sung $\Delta t$\\
\textbf{Nachteil:} Aufwendige Matrixinversion n�tig.

\subsubsection{Crank Nicolson -Verfahren (gemischtes Verfahren)}

Die Idee des Verfahrens von Crank-Nicolson ist es die Approximationen\\

$\boxed{\frac{\tilde{u}(x,t+\Delta t) - \tilde{u}(x,t)}{\Delta t} = 
\frac{\tilde{u}(x+\Delta x, t)-2\tilde{u}(x,y) + \tilde{u}( x - \Delta x, t )} {\Delta x^2}}$\\

und\\

$\boxed{\frac{\tilde{u}(x,t) - \tilde{u}(x,t -\Delta t)}{\Delta t} = 
\frac{\tilde{u}(x+\Delta x, t)-2\tilde{u}(x,y) + \tilde{u}( x - \Delta x, t )} {\Delta x^2}}$\\

zu mitteln. Mit dieser Idee geht das stetige Problem in folgendes diskretes Problem �ber:

$-r \tilde{u}_{j-1,k+1} + (2+2r)\tilde{u}_{j,k+1} - r \tilde{u}_{j+1,k+1} = r
\tilde{u}_{j-1,k+1} + (2-2r)\tilde{u}_{j,k+1} + r \tilde{u}_{j+1,k+1} $



\begin{itemize}
\item Initialisierung, Randbedingung: $\tilde{u}_{j,0}=f(j/n)$ \qquad $\tilde{u}_{0,k}=\tilde{u}_{n,k}=0$
\item Approximationsmatrizen:\\
$F^{(n)}=E^{(n)}+I=tridiag_{n-1}(-r,2+2r,-r)=\begin{bmatrix}
2+2r& -r	& 0		& 0 	&\cdots\\
-r	& 2+2r  & -r	& 0		&\cdots\\
0	& -r	& 2+2r 	& -r 	&\cdots\\
0	& 0		& -r	& 2+2r 	&\cdots\\
\vdots&	\vdots&\vdots&\vdots&\ddots	
\end{bmatrix}$\\
$G^{(n)}=C^{(n)}+I=tridiag_{n-1}(~r,~2-2r,~r~)=\begin{bmatrix}
2-2r& r		& 0		& 0 	&\cdots\\
r	& 2-2r  & r		& 0		&\cdots\\
0	& r		& 2-2r 	& r 	&\cdots\\
0	& 0		& r		& 2-2r 	&\cdots\\
\vdots&	\vdots&\vdots&\vdots&\ddots	
\end{bmatrix}$ 
\item Einen Schritt berechnen: $\tilde{u}^{(k+1)}=\left\{F^{(n)}\right\}^{-1} \cdot G^{(n)}\cdot \tilde{u}^{(k)}$
\item k-Schritte berechnen: $\tilde{u}^{(k)}=\left(\left\{F^{(n)}\right\}^{-1} \cdot G^{(n)}\right)^{k}\cdot \tilde{u}^{(0)}$
\end{itemize}

\subsection{(FDM f�r Hyperbolische PDGL)}

$$u_{tt}=u_{xx} = homogen$$
$$u_{tt} -u_{xx}= v(x,t) = inhomogen$$

$$\Longrightarrow u_{j,k+1}=r^2 u_{j-1,k} 2(1-r^2)u_{j,k}+ r^2
u_{j+1,k}-u_{j,k-1}$$
\subsubsection{Anfangsbedingungen}
$$\tilde{u}_{j,0} = f(j\Delta x); u_{j,1}= f(j\Delta x) + g(j\Delta x)\Delta t$$
(f(x)= Anfangsfunktion; g(x)=Ableitung; $r = \frac{\Delta t}{\Delta x}$)

\subsubsection{Transportgleichung}
$$u_x(x,t) + u_t(x, t) = 0; u(x,0)=P(x-t)$$
$$\frac{u(x,t+\Delta t)-u(x,t)}{\Delta t} = \frac{u(x,t) - u(x-\Delta x,
t)}{\Delta x}$$

\subsection{FVM (Finite Volumen Methode)}
$\Delta u=0$\qquad in \quad$\Omega$\\
$u(x,y)=f(x,y)$ \qquad auf \quad$\partial\Omega$\\
Der Satz von Gauss sagt: $\boxed{\oint\limits_{\Gamma}{\Delta u(x,y) dx dy}=\int\limits_{\Gamma}{\mathrm{div}~\mathrm{grad}~ u(x,y) dx dy}=\oint\limits_{\partial\Gamma}{\mathrm{grad}~ u(x,y) d\vec{n}}}$\\


Wobei der Randnormalvektor $\vec{n}$ immer senkrecht gegen das Aussengebiet $\Gamma$ gerichtet wird.\\

$\Rightarrow$ $\oint\limits_{\Gamma}{\Delta u(x,y) dx dy}=\oint\limits_{\partial\Gamma}{\mathrm{grad}~ u(x,y) d\vec{n}}=0$

\begin{minipage}{4cm}
	\includegraphics[width=4cm]{Content/Numerik/FVMPrinzip.png}
\end{minipage}
\hfill
\begin{minipage}{14cm}
	$\frac{u(P_E)-u(P_P)}{h}\cdot h+\frac{u(P_N)-u(P_P)}{h}\cdot h+\frac{u(P_W)-u(P_P)}{h}\cdot h+\frac{u(P_S)-u(P_P)}{h}\cdot h\approx 0$\\
	
	$\Rightarrow\tilde{u}(P_E)+\tilde{u}(P_N)+\tilde{u}(P_W)+\tilde{u}(P_S)+4\cdot\tilde{u}(P_P)=0$
\end{minipage}

\textbf{Vorteile:}\\
\begin{itemize}
\item Man kann mit Flussgr�ssen un Bilanzen rechnen, dadurch kann der Laplace-Operator $(\Delta)$ verzichtet werden und somit die aufwendige Mathematik umgangen werden.
\item Es kann mit komplizierten Geometrien gerechnet werden.  
\end{itemize}

\begin{minipage}{4cm}
	\includegraphics[width=4cm]{Content/Numerik/FVMPrinzip.png}
\end{minipage}
\hfill
\begin{minipage}{14cm}
	$\frac{u(P_E)-u(P_P)}{h}\cdot h+\frac{u(P_N)-u(P_P)}{h}\cdot h+\frac{u(P_W)-u(P_P)}{h}\cdot h+\frac{u(P_S)-u(P_P)}{h}\cdot h\approx 0$\\
	
	$\Rightarrow\tilde{u}(P_E)+\tilde{u}(P_N)+\tilde{u}(P_W)+\tilde{u}(P_S)+4\cdot\tilde{u}(P_P)=0$
\end{minipage}




\begin{minipage}{6cm}
	\includegraphics[width=6cm]{Content/Numerik/FVM1.png}
\end{minipage}
\hfill
\begin{minipage}{12cm}
\textbf{Vorgehen bei der Berechnung:}\\
\begin{enumerate}
\item Punkte $P_1,\ldots,P_n$ w�hlen.
\item Aufteilen des Bereichs in kleine Teilbereiche, z.B. durch Mittelsenkrechte
\item Rand diskretisieren.\\
\end{enumerate}

F�r P1-Zelle:\quad $\frac{\tilde{u}(P_2)-\tilde{u}(P_1)}{\delta_{1,2}}\cdot\lambda_{1,2}+\frac{\tilde{u}(P_3)-\tilde{u}(P_1)}{\delta_{1,3}}\cdot\lambda_{1,3}+\frac{\tilde{u}(R_1)-\tilde{u}(P_1)}{\delta_1}\cdot\lambda_1$\\

F�r P2-Zelle:\quad $\frac{\tilde{u}(P_1)-\tilde{u}(P_2)}{\delta_{1,2}}\cdot\lambda_{1,2}+\frac{\tilde{u}(P_3)-\tilde{u}(P_2)}{\delta_{2,3}}\cdot\lambda_{2,3}+\frac{\tilde{u}(R_2)-\tilde{u}(P_2)}{\delta_2}\cdot\lambda_2$\\

F�r P3-Zelle:\quad $\frac{\tilde{u}(P_2)-\tilde{u}(P_3)}{\delta_{2,3}}\cdot\lambda_{2,3}+\frac{\tilde{u}(P_1)-\tilde{u}(P_3)}{\delta_{1,3}}\cdot\lambda_{1,3}+\frac{\tilde{u}(R_3)-\tilde{u}(P_3)}{\delta_3}\cdot\lambda_3$\\
\end{minipage}

\begin{minipage}{6cm}
	\includegraphics[width=6cm]{Content/Numerik/FVM2.png}
\end{minipage}
\hfill
\begin{minipage}{12cm}

 $\frac{\tilde{u}(P_E)-\tilde{u}(P_N)}{1/4\cdot\sqrt{2}}\cdot\frac{\sqrt{2}}{2}+\frac{\tilde{u}(P_W)-\tilde{u}(P_N)}{1/4\cdot\sqrt{2}}\cdot\frac{\sqrt{2}}{2}+\frac{\tilde{u}(P_N)-\tilde{u}(R_N)}{1/4}\cdot 1=0$\\
 
 $(\tilde{u}_E-\tilde{u}_N)\cdot 2 + (\tilde{u}_W-\tilde{u}_N)\cdot 2 + (1/2-\tilde{u}_N)\cdot 4=0$\\
 
 $0\cdot\tilde{u}_S+2\cdot\tilde{u}_E+2\cdot\tilde{u}_W-8\cdot\tilde{u}_N+2=0$
 

\end{minipage}



\section{FEM}

Der Vektorraum $\mathbb{V}$ hat undendlich viele Dimensionen. Falls wir n unabh�ngige Funktionen $v_1,\ldots,v_n$ w�hlen, dann spannen die Funktionen $a_1\cdot v_1(x)+\ldots+a_n\cdot v_n(x)$ einen n dimensionalen Teilraum $\mathbb{V}^{(n)}$  von $\mathbb{V}$ auf. Dabei gilt:\\

$\boxed{\tilde{u}^{(n)}=a_1\cdot v_1(x)+\ldots+a_n\cdot v_n(x)}$
\subsection{Das Verfahren von Ritz}
\textbf{Ritzsche Matrize: }
$R^{(n)}=\begin{bmatrix}
	R_{1,1}& R_{1,2}&\cdots\\
	R_{2,1}& R_{2,2}&\cdots\\
	\vdots & \vdots &\ddots\\
\end{bmatrix}$ \qquad mit \qquad $R_{j,k}^{(n)}=\int\limits_{0}^{1}{v_j'(x)\cdot v_k'(x) dx}$\\
\textbf{Ritzscher Vektor: } 
$r^{(n)}=\begin{bmatrix}
	r_1\\
	r_2\\
	\vdots\\
\end{bmatrix}$ \qquad mit \qquad $r_{k}^{(n)}=\int\limits_{0}^{1}{f(x)\cdot v_k(x) dx}$\\

\textbf{L�sung nach Ritz:} $R^{(n)}\cdot a=r^{(n)}\qquad \Rightarrow \qquad a=\left\{R^{(n)}\right\}^{-1}\cdot r^{(n)}$
\subsection{Das Verfahren von Galerkin}
\textbf{Galerksche Matrize: }
$G^{(n)}=\begin{bmatrix}
	G_{1,1}& G_{1,2}&\cdots\\
	G_{2,1}& G_{2,2}&\cdots\\
	\vdots & \vdots &\ddots\\
\end{bmatrix}$ \qquad mit \qquad $G_{j,k}^{(n)}=\int\limits_{0}^{1}{v_j''(x)\cdot v_k(x) dx}$\\
\textbf{Galerkscher Vektor: } 
$g^{(n)}=\begin{bmatrix}
	g_1\\
	g_2\\
	\vdots\\
\end{bmatrix}$ \qquad mit \qquad $g_{k}^{(n)}=\int\limits_{0}^{1}{f(x)\cdot v_k(x) dx}$\\

\textbf{L�sung nach Galerkin:} $G^{(n)}\cdot a+g^{(n)}=0\qquad \Rightarrow \qquad a=\textcolor{red}{\mathbf{-}}\left\{G^{(n)}\right\}^{-1}\cdot g^{(n)}$


\subsection{Gewichtete Residuen (Bereichskollokation)}

Gewichtungsfunktionen: $\{w_1(x),\ldots,w_n(x)\}$

\textbf{Matrize (gewichtete Residuen): }
$M^{(n)}=\begin{bmatrix}
	M_{1,1}& M_{1,2}&\cdots\\
	M_{2,1}& M_{2,2}&\cdots\\
	\vdots & \vdots &\ddots\\
\end{bmatrix}$ \qquad mit \qquad $M_{j,k}^{(n)}=\int\limits_{0}^{1}{v_j''(x)\cdot w_k(x) dx}$\\
\textbf{Vektor (gewichtete Residuen): } 
$m^{(n)}=\begin{bmatrix}
	m_1\\
	m_2\\
	\vdots\\
\end{bmatrix}$ \qquad mit \qquad $m_{k}^{(n)}=\int\limits_{0}^{1}{f(x)\cdot w_k(x) dx}$\\

\textbf{L�sung der gewichteten Residuen:} $M^{(n)}\cdot a+m^{(n)}=0\qquad \Rightarrow \qquad a=\textcolor{red}{\mathbf{-}}\left\{M^{(n)}\right\}^{-1}\cdot m^{(n)}$

\subsection{Punktkollokation}
Im Sinne einer Punktkollokation werden n St�tzstellen im Intervall von [0,1] gew�hlt.\\

$\begin{bmatrix}
	v_1''(x_1)& v_2''(x_1)&\cdots\\
	v_1''(x_2)& v_2''(x_2)&\cdots\\
	\vdots& \vdots&\ddots
\end{bmatrix}\cdot
\begin{bmatrix}
a_1\\
a_2\\
\vdots
\end{bmatrix}
=\begin{bmatrix}
-f(x_1)\\
-f(x_2)\\
\vdots
\end{bmatrix}$\qquad Das Gleichungssystem nach a aufl�sen

\subsection{Das Verfahren von Gauss (MSE)}

\textbf{Gausscher Matrize: }
$Q^{(n)}=\begin{bmatrix}
	Q_{1,1}& Q_{1,2}&\cdots\\
	Q_{2,1}& Q_{2,2}&\cdots\\
	\vdots & \vdots &\ddots\\
\end{bmatrix}$ \qquad mit \qquad $Q_{j,k}^{(n)}=\int\limits_{0}^{1}{v_j''(x)\cdot v_k''(x) dx}$\\
\textbf{Gausscher Vektor: } 
$q^{(n)}=\begin{bmatrix}
	q_1\\
	q_2\\
	\vdots\\
\end{bmatrix}$ \qquad mit \qquad $q_{k}^{(n)}=\int\limits_{0}^{1}{f(x)\cdot v_k''(x) dx}$\\

\textbf{L�sung nach Gauss:} $Q^{(n)}\cdot a+q^{(n)}=0\qquad \Rightarrow \qquad a=\textcolor{red}{\mathbf{-}}\left\{Q^{(n)}\right\}^{-1}\cdot q^{(n)}$