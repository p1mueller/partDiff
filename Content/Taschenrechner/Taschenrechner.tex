\section{Taschenrechnerbefehle}
\renewcommand{\arraystretch}{1.5}
\subsection{FDM}
\begin{tabularx}{\textwidth}{l X}
    \hline
    \textbf{\texttt{fdm1d(dx,x0,xn,a,b,f,g,h,p)}}              & Löst $g(x)\cdot u''(x) + h(x)\cdot u'(x)+p(x)\cdot u(x)=f(x)$ im Gebiet $\Omega = [x_0, x_n]$ mit den Randbedingungen $u(x_0)=a$ und $u(x_n)=b$  mittels FDM mit Schrittweite $dx$. Es wird die \textbf{zentrale Differenz} verwendet.                                                                                                                                                    \\\hline
    \textbf{\texttt{fdmelstar(h,x0,xn,y0,yn,f,g,c)}}           & Löst $\triangle u(x,y) + c\cdot u(x,y) = f(x,y)$ im Gebiet $\Omega = [x_0, x_n]\times [y_0,y_n]$ mit den Randbedingungen $u(x,y)=g(x,y)$ mittels FDM mit Schrittweite $h$. Es wird der \textbf{Five-Point-Star-Operator} verwendet.                                                                                                                                                       \\\hline
    \textbf{\texttt{fdmpbexplicit(dx,dt,x0,xn,t0,tn,a,f,g,h)}} & Löst $u_t(x,t) = a\cdot u_{xx}(x,t)$ im Gebiet $\Omega = [x_0, x_n]\times [t_0,t_n]$ mit den Randbedingungen $u(x,t_0)=f(x)$, $u(x_0,t)=g(t)$ und $u(x_n,t)=h(t)$ mittels FDM mit Schrittweiten $dx$, $dt$. Es wird das \textbf{explizite Verfahren nach Richard} verwendet.                                                                                                              \\\hline
    \textbf{\texttt{fdmpbrichard(dx,dt,x0,xn,t0,tn,a,f,g,h)}}  & Löst $u_t(x,t) = a\cdot u_{xx}(x,t)$ im Gebiet $\Omega = [x_0, x_n]\times [t_0,t_n]$ mit den Randbedingungen $u(x,t_0)=f(x)$, $u(x_0,t)=g(t)$ und $u(x_n,t)=h(t)$ mittels FDM mit Schrittweiten $dx$, $dt$. Es wird das \textbf{implizite Verfahren nach Richard} verwendet.                                                                                                              \\\hline
    \textbf{\texttt{fdmpbcrank(dx,dt,x0,xn,t0,tn,a,f,g,h)}}    & Löst $u_t(x,t) = a\cdot u_{xx}(x,t)$ im Gebiet $\Omega = [x_0, x_n]\times [t_0,t_n]$ mit den Randbedingungen $u(x,t_0)=f(x)$, $u(x_0,t)=g(t)$ und $u(x_n,t)=h(t)$ mittels FDM mit Schrittweiten $dx$, $dt$. Es wird das \textbf{Crank-Nicolson Verfahren} verwendet.                                                                                                                      \\\hline
    \textbf{\texttt{fdmhbdown(dx,dt,x0,xn,t0,tn,f)}}           & Löst $u_x(x,t)+u_t(x,t)=0$ im Gebiet $\Omega = [x_0, x_n]\times [t_0,t_n]$ mit den Randbedingungen $u(x,t_0)=f(x)$ und $u(x_0,t)=u(x_n,t)=0$ mittels FDM mit Schrittweiten $dx$, $dt$. Es wird das \textbf{Downwind Verfahren} verwendet. {\color{red}Achtung:} die letzte Zeile der Ausgabe entspricht der $x$-Koordinate. Das finale Resultat befindet sich in der 2. letzen Zeile.     \\\hline
    \textbf{\texttt{fdmhbup(dx,dt,x0,xn,t0,tn,f)}}             & Löst $u_x(x,t)+u_t(x,t)=0$ im Gebiet $\Omega = [x_0, x_n]\times [t_0,t_n]$ mit den Randbedingungen $u(x,t_0)=f(x)$ und $u(x_0,t)=u(x_n,t)=0$ mittels FDM mit Schrittweiten $dx$, $dt$. Es wird das \textbf{Upwind Verfahren} verwendet. {\color{red}Achtung:} die letzte Zeile der Ausgabe entspricht der $x$-Koordinate. Das finale Resultat befindet sich in der 2. letzen Zeile.       \\\hline
    \textbf{\texttt{fdmhbcenter(dx,dt,x0,xn,t0,tn,f)}}         & Löst $u_x(x,t)+u_t(x,t)=0$ im Gebiet $\Omega = [x_0, x_n]\times [t_0,t_n]$ mit den Randbedingungen $u(x,t_0)=f(x)$ und $u(x_0,t)=u(x_n,t)=0$ mittels FDM mit Schrittweiten $dx$, $dt$. Es wird das \textbf{zentrale Verfahren} verwendet. {\color{red}Achtung:} die letzte Zeile der Ausgabe entspricht der $x$-Koordinate. Das finale Resultat befindet sich in der 2. letzen Zeile.     \\\hline
    \textbf{\texttt{fdmhblax(dx,dt,x0,xn,t0,tn,f)}}            & Löst $u_x(x,t)+u_t(x,t)=0$ im Gebiet $\Omega = [x_0, x_n]\times [t_0,t_n]$ mit den Randbedingungen $u(x,t_0)=f(x)$ und $u(x_0,t)=u(x_n,t)=0$ mittels FDM mit Schrittweiten $dx$, $dt$. Es wird das \textbf{Lax-Wendroff Verfahren} verwendet. {\color{red}Achtung:} die letzte Zeile der Ausgabe entspricht der $x$-Koordinate. Das finale Resultat befindet sich in der 2. letzen Zeile. \\\hline
    \textbf{\texttt{fdmhbleap(dx,dt,x0,xn,t0,tn,f,g)}}         & Löst $u_{xx}(x,t)=u_{tt}(x,t)$ im Gebiet $\Omega = [x_0, x_n]\times [t_0,t_n]$ mit den Randbedingungen $u(x,t_0)=f(x)$ und $u_t(x,t_0)=g(x)$ mittels FDM mit Schrittweiten $dx$, $dt$. Es wird das \textbf{Leapfrog Verfahren} verwendet.                                                                                                                                                 \\\hline
\end{tabularx}
\clearpage

\subsection{FEM}
\begin{tabularx}{\textwidth}{l X}
    \hline
    \textbf{\texttt{femritz(x0,xn,v,f)}}          & Löst $u''(x)+f(x)=0$ im Gebiet $\Omega = [x_0, x_n]$ mit den Randbedingungen $u(x_0)=u(x_n)=0$ mittels FEM anhand den Stützfunktionen in $v$. $v$ ist ein Spaltenvektor mit den Stützfunktionen. Es wird die \textbf{Methode von Ritz} verwendet.                                          \\\hline
    \textbf{\texttt{femgalerkin(x0,xn,v,f,g,h)}}  & Löst $u''(x)+g(x)\cdot u'(x)+h(x)\cdot u(x)+f(x)=0$ im Gebiet $\Omega = [x_0, x_n]$ mit den Randbedingungen $u(x_0)=u(x_n)=0$ mittels FEM anhand den Stützfunktionen in $v$. $v$ ist ein Spaltenvektor mit den Stützfunktionen. Es wird die \textbf{Methode von Galerkin} verwendet.       \\\hline
    \textbf{\texttt{femlinear(h,x0,xn,a,b,f)}}    & Löst $u''(x)+f(x)=0$ im Gebiet $\Omega = [x_0, x_n]$ mit den Randbedingungen $u(x_0)=a$ und $u(x_n)=b$ mittels FEM mit \textbf{linearen Stützfunktionen mit Breite $h$}.                                                                                                                   \\\hline
    \textbf{\texttt{femlinearnu(steps,a,b,f)}}    & Löst $u''(x)+f(x)=0$ im Gebiet $\Omega = [x_0, x_n]$ mit den Randbedingungen $u(x_0)=a$ und $u(x_n)=b$ mittels FEM mit \textbf{linearen Stützfunktionen mit dynamischen Breiten}. Die Intervalle der Stützfunktionen ist im Zeilenvektor \texttt{steps} definiert (z.B. [0, 1/6, 1/2, 1]). \\\hline
    \textbf{\texttt{femquadratic(h,x0,xn,a,b,f)}} & Löst $u''(x)+f(x)=0$ im Gebiet $\Omega = [x_0, x_n]$ mit den Randbedingungen $u(x_0)=a$ und $u(x_n)=b$ mittels FEM mit \textbf{quadratischer Stützfunktionen mit Breite $h$}.                                                                                                              \\\hline
\end{tabularx}
\renewcommand{\arraystretch}{1.0}