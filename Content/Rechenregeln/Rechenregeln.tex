\section{Mathe Grundlagen}

\subsection{Partialbruchzerlegung}
\input{Content/Rechenregeln/PBZ}

\subsection{Trigonometrie}
\input{Content/Rechenregeln/Trigo}

\subsection{Taylor Polynom}
$f(x_0+h)=f(x_0) + f'(x_0)h + \frac{f''(x_0)}{2}h^2 + \frac{f'''(x_0)}{3!}h^3 + \ldots + \frac{f^{(n)}(x_0)}{n!}h^n + R_n(x_0, h)$

\subsection{Integralrechnung}
\begin{tabbing}
	xxxxxxxxxxxxxxxxxxxxxxxxxxxxxxx \= xxx \= xxxxxxxxxxxxxxxxxxxxxxxxxxxxxxxxxxxxxxxxxxxxxxxxxxxxxxx\kill
	Integration\>\>$A=\int\limits_{a}^{b}{f(t)dt}=\left[F(t)\right]_a^b=F(b)-F(a)$\\[0.2cm]
	Linearität\>
	$\qquad\int{f(\alpha x+\beta )dx=\frac{1}{\alpha}\cdot F(\alpha x+
			\beta)+C}$\\[0.2cm]
	Partielle Integration\>
	$\qquad\int\limits_a^b{\underset{\Uparrow}{u}'(x)\cdot \underset{\Downarrow}{v}(x)dx}=\biggl[ u(x)\cdot v(x) \biggr]_a^b
		-\int\limits_a^b{u(x)\cdot v'(x)dx}$\\[0.2cm]
	Substitution (Rationalisierung)\>
	$\qquad t=\tan\frac{x}{2}, \qquad dx=\frac{2dt}{1+t^2} \qquad
		\sin  x=\frac{2t}{1+t^2} \qquad \cos x=\frac{1-t^2}{1+t^2}
		\quad\int{R(\sin(x)\cos(x))dx}$\\
	Allgemeine Substitution \> \>
	$\int\limits_{a}^{b}{f(x)dx}=\int\limits_{g^{-1}(a)}^{g^{-1}(b)}{f(g(t))\cdot
			g'(t)dt}\qquad t=g^{-1}(x)\qquad  \fbox{x=g(t)}\qquad dx=g'(t)\cdot dt$\\
	Logarithmische Integration \>\>
	$ \int{\frac{f'(x)}{f(x)}dx}=\ln|f(x)|+C	\qquad{(f(x)\neq 1)}$\\[0.2cm]
	Spezielle Form des Integranden \>\>
	$\int{f'(x)\cdot (f(x))^{\alpha} dx}= f(x)^{\alpha +1}\cdot
		\frac{1}{\alpha+1}+C \qquad{(\alpha \neq -1)}$\\
	Differentiation\>\>
	$\int \limits ^{b} _{a} {f'(t)dt}=f(b)-f(a)$\qquad
	$\frac{d}{dx} \int \limits ^{x} _{1} {f(t)dt}=f(x)$
\end{tabbing}
%\subsubsection{Einige unbestimmte Integrale}
%\includegraphics[width=19cm]{Content/Rechenregeln/integrale.png} 

$\int \mathrm dx = x + C$\\
$\int x^{\alpha} \mathrm dx = \frac{x^{\alpha + 1}}{\alpha + 1} + C$, $x \in
	\mathbb{R}^+$, $\alpha \in \mathbb{R} \textbackslash \{-1\}$\\
$\int \frac{1}{x} \mathrm dx = \ln \lvert x \lvert + C$, $x \neq 0$\\
$\int e^x \mathrm dx = e^x + C$\\
$\int a^x \mathrm dx = \frac{a^x}{\ln(a)} + C$, $a \in \mathbb{R}^+
	\textbackslash \{1\}$\\
$\int \sin x \mathrm dx = - \cos x + C$\\
$\int \cos x \mathrm dx = \sin x + C$\\
$\int \frac{\mathrm dx}{\sin^2 x} = -\cot(x) + C$, $x \neq k \pi$ mit $k \in
	\mathbb{Z}$\\
$\int \frac{\mathrm dx}{\cos^2 x} = \tan(x) + C$, $x \neq \frac{\pi}{2} + k \pi$
mit $k \in \mathbb{Z}$\\
$\int \sinh(x) \mathrm dx = \cosh(x) + C$\\
$\int \cosh(x) \mathrm dx = \sinh(x) + C$\\
$\int \frac{\mathrm dx}{\sinh^2 x} = -\coth(x) + C$, $x \neq 0$\\
$\int \frac{\mathrm dx}{\cosh^2 x} = \tanh(x) + C$, $x \neq 0$\\
$\int \frac{\mathrm dx}{ax + b} = \frac{1}{a} \ln \lvert ax + b \lvert + C$, $a
	\neq 0$, $x \neq - \frac{b}{a}$\\
$\int \frac{\mathrm dx}{a^2x^2 + b^2} = \frac{1}{ab} \arctan(\frac{a}{b} x) +
	C$, $a \neq 0$, $x \neq - \frac{b}{a}$ $x \neq -\frac{b}{a}$\\
$\int \frac{\mathrm dx}{a^2x^2 - b^2} = \frac{1}{2ab} \ln \lvert \frac{ax -
		b}{ax + b} \lvert + C$, $a \neq 0$, $x \neq - \frac{b}{a}$ $x \neq
	-\frac{b}{a}$\\
$\int \sqrt{a^2 x^2 + b^2} \mathrm dx = \frac{x}{2} \sqrt{a^2 x^2 + b^2} +
	\frac{b^2}{2a} \ln(ax + \sqrt{a^2 x^2 + b^2}) + C$, $a \neq 0$, $b \neq 0$\\
$\int \sqrt{a^2 x^2 - b^2} \mathrm dx = \frac{x}{2} \sqrt{a^2 x^2 - b^2} +
	\frac{b^2}{2a} \ln \lvert ax + \sqrt{a^2 x^2 - b^2} \lvert + C$, $a \neq 0$, $b
	\neq 0$, $a^2 x^2 \geq b^2$\\
$\int \sqrt{b^2 - a^2x^2} \mathrm dx = \frac{x}{2} \sqrt{b^2 - a^2x^2} +
	\frac{b^2}{2a} \arcsin(\frac{a}{b} x) + C)$, $a \neq 0$, $b \neq 0$,
$a^2 x^2 \geq b^2$\\
$\int \frac{\mathrm dx}{\sqrt{a^2 x^2 + b^2}} = \frac{1}{a} \ln(ax +
	\sqrt{a^2 x^2 + b^2}) + C$, $a \neq 0$, $b \neq 0$\\
$\int \frac{\mathrm dx}{\sqrt{a^2 x^2 - b^2}} = \frac{1}{a} \ln(ax +
	\sqrt{a^2 x^2 - b^2}) + C$, $a \neq 0$, $b \neq 0$ $a^2 x^2 > b^2$\\
$\int \frac{\mathrm dx}{\sqrt{b^2 - a^2 x^2}} = \frac{1}{a}
	\arcsin(\frac{a}{b} x) + C$, $a \neq 0$, $b \neq 0$, $a^2 x^2 < b^2$\\
Die Integrale $\int \frac{dx}{X}$, $\int \sqrt{X} \; dx$, $\int
	\frac{dx}{\sqrt{X}}$ mit $X = ax^2 + 2bx + c$, $a \neq 0$, werden durch die
Umformung $X = a \left(x + \frac{b}{a}\right)^2 + \left(c -
	\frac{b^2}{a}\right)$ und die Substitution $t = x + \frac{b}{a}$ in die
Integrale 15. bis 22. transformiert.\\
$\int \frac{x \; \mathrm dx}{X}= \frac{1}{2a} \ln \lvert X \lvert - \frac{b}{a}
	\int \frac{dx}{X}$, $a \neq 0$, $X = ax^2 + 2bx + c$\\
$\int \sin^2(ax) \; \mathrm dx = \frac{x}{2} - \frac{1}{4a} \cdot \sin(2ax) +
	C$, $a \neq 0$\\
$\int \cos^2(ax) \; \mathrm dx = \frac{x}{2} + \frac{1}{4a} \cdot \sin(2ax) +
	C$, $a \neq 0$\\
$\int \sin^n(ax) \; \mathrm dx = \frac{\sin^{n-1}(ax) \cdot \cos(ax)}{na} +
	\frac{n-1}{n} \int \sin^{n-2}(ax) \mathrm dx$, $n \in \mathbb N$, $a \neq 0$\\
$\int \cos^n(ax) \; \mathrm dx = \frac{cos^{n-1}(ax) \cdot \sin(ax)}{na} +
	\frac{n-1}{n} \int \cos^{n-2}(ax) \mathrm dx$, $n \in \mathbb N$, $a \neq 0$\\
$\int \frac{dx}{\sin (ax)} = \frac{1}{a} \ln \lvert \tan(\frac{ax}{2}) \lvert +
	C$, $a \neq 0$, $x \neq k \frac{\pi}{a}$ mit $k \in \mathbb Z$\\
$\int \frac{dx}{cos (ax)} = \frac{1}{a} \ln \lvert \tan(\frac{ax}{2} +
	\frac{\pi}{4}) \lvert + C$, $a \neq 0$, $x \neq \frac{\pi}{2a} + k
	\frac{\pi}{a}$ mit $k \in \mathbb Z$\\
$\int \tan(ax) \mathrm dx = - \frac{1}{a} \ln \lvert \cos(ax) \lvert + C$, $a
	\neq 0$, $x \neq \frac{\pi}{2a} + k \frac{\pi}{a}$ mit $k \in \mathbb Z$\\
$\int \cot(ax) \mathrm dx = \frac{1}{a} \ln \lvert \sin(ax) \lvert + C$, $a
	\neq 0$, $x \neq k \frac{\pi}{a}$ mit $k \in \mathbb Z$\\
$\int x^n \sin(ax) \mathrm dx = - \frac{x^n}{a} \cos(ax) + \frac{n}{a} \int
	x^{n-1} \cos(ax) \mathrm dx$, $n \in \mathbb N$, $a \neq 0$\\
$\int x^n \cos(ax) \mathrm dx = \frac{x^n}{a} \sin(ax) - \frac{n}{a} \int
	x^{n-1} \sin(ax) \mathrm dx$, $n \in \mathbb N$, $a \neq 0$\\
$\int x^n e^{ax} \mathrm dx = \frac{1}{a} x^n e^{ax} - \frac{n}{a} \int
	x^{n-1} e^{ax} \mathrm dx$, $n \in \mathbb N$, $a \neq 0$\\
$\int e^{ax} \sin(bx) \mathrm dx = \frac{e^{ax}}{a^2 + b^2} (a \cdot \sin(bx) -
	b \cdot \cos(bx)) + C$, $a \neq 0$, $b \neq 0$ \\
$\int e^{ax} \cos(bx) \mathrm dx = \frac{e^{ax}}{a^2 + b^2} (a \cdot \cos(bx) +
	b \cdot \sin(bx)) + C$, $a \neq 0$, $b \neq 0$ \\
$\int \ln(x) \mathrm dx = x(\ln(x) - 1) + C$, $x \in \mathbb R^+$\\
$\int x^\alpha \cdot \ln(x) \mathrm dx = \frac{x^{\alpha + 1}}{(\alpha + 1)^2}
	[(\alpha + 1) \ln(x) - 1] + C$, $x \in \mathbb R^+$, $\alpha \in \mathbb R
	\textbackslash \{-1\}$\\


\subsection{Differentialgleichungen}
\input{Content/Rechenregeln/DGL}
\subsection{Differential-Rechnung}
$f'(x_0)=\lim\limits_{\Delta x\rightarrow 0}
	\frac{f(x_0+\Delta)x-f(x_0)}{\Delta x}$\\
\begin{tabular}{llll}
	Kettenregel:                   & $f\big(g(x)\big)'$ & $=$                                 & $g'(x)\cdot f'\big(g(x)\big)$
	oder $\frac{d f(g(x))}{dx} = f'(g(x)) \cdot g'(x)$                                                                        \\[0.1cm] Produktregel:	&
	$\left(f(x)\cdot g(x)\right)'$ & $=$                & $f'(x)\cdot g(x) + f(x)\cdot g'(x)$                                 \\[0.1cm] Quotientenregel:& $\frac{f(x)}{g(x)}$ &$=$ & $\frac{f'(x)g(x)-f(x)g'(x)}{g^2(x)}$\\
\end{tabular}

\subsection{Diverses}
\begin{minipage}[t]{9.5cm}
	\subsubsection{Quadratische Lösungsformel}
	$ax^2+bx+c=0\quad\Rightarrow\quad x_{1,2}=\frac{-b\pm\sqrt{b^2-4ac}}{2a}$
\end{minipage}
\hfill
\begin{minipage}[t]{9.5cm}
	\subsubsection{Determinanten}
	$\det\left(
		\begin{bmatrix}
				a_{11} & a_{12} \\
				a_{21} & a_{22} \\
			\end{bmatrix}\right)=
		\begin{vmatrix}
			a_{11} & a_{12} \\
			a_{21} & a_{22} \\
		\end{vmatrix}=a_{11}a_{22}-a_{12}a_{21}$
\end{minipage}\\

\begin{minipage}[t]{9.5cm}
	\subsubsection{Matrizeninversion}
	$A=\begin{bmatrix}
			a_{11} & a_{12} \\
			a_{21} & a_{22} \\
		\end{bmatrix}\quad\Rightarrow\quad
		A^{-1}=\frac{1}{det(A)}
		\begin{bmatrix}
			a_{22}  & -a_{12} \\
			-a_{21} & a_{11}  \\
		\end{bmatrix}$
\end{minipage}
\hfill
\begin{minipage}[t]{9.5cm}
	\subsubsection{Eigenwerte/ Eigenvektoren}
	Eigenwert: $\det(A-\lambda I)\quad\Rightarrow\quad \lambda i$\\
	Eigenvektor: $(A-\lambda_i I)v=0\quad\Rightarrow\quad v_i\qquad(\text{Für jedes }\lambda_i)$\\
	Definition: $ A\cdot \underline{v} = \lambda \cdot \underline{v} $
\end{minipage}
\clearpage

% \begin{minipage}[t]{9.5cm}
% 	\subsubsection{TI-89}
% 		\textbf{Gleichung für mehrere Werte}\\
% 		$(3x+y^2) \mid x=1 \text{ and } y=2 \to Resultat$\\
% 		\textbf{Matrizeneditor}
% 		\begin{itemize}
% 		  \item APPS / Data/Matrix Editor
% 		  \item New
% 		  \item Type: Matrix
% 		  \item Variable, Row, Column definieren
% 		  \item Werte eingeben
% 		\end{itemize}

% 		\textbf{Gespeicherte Variabeln löschen}
% 		\begin{itemize}
% 		\item Explorer: 2nd / VAR-LINK
% 		\item Variable anwählen
% 		\item löschen: DEL
% 		\item Löschen Bestätigen: ENTER
% 		\end{itemize}

% \end{minipage}

