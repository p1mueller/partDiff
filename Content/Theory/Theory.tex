\section{Theory}

\subsection{Begriffe und Klassifikation}

\subsubsection{Ordnung}

Wie bei gewöhnlichen Differentialgleichungen ist die Ordnung
die höchste Ableitung der unbekannten Funktion, die in der
Differentialgleichung vorkommt.\\

\textbf{PDGL 1. Ordnung: } \qquad$F\biggl(x_1,\dots,x_n, u, \frac{\partial u}{\partial x_1},\dots,\frac{\partial u}{\partial x_n}\biggr)$

Sie kann durch Substitution $\frac{\partial u}{\partial x_i}\to p_i$ durch  $F(x_1,\dots,x_n,u,p_1,\dots,p_n),$
ausgedrückt werden.\\

\textbf{PDGL 2. Ordnung: } \qquad $F\biggl(x_1,\dots,x_n,u,
\frac{\partial u}{\partial x_1},\dots,\frac{\partial u}{\partial x_n},
\frac{\partial^2 u}{\partial x_1^2},\dots,\frac{\partial^2 u}{\partial x_n^2}\biggr)$

Sie kann durch Substitution $\frac{\partial u}{\partial x_i}\to p_i,~\frac{\partial^2 u}{\partial x_i\partial x_j}\to t_{ij}$ durch
$F(x_1,\dots,x_n,u,p_1,\dots,p_n,t_{11},t_{12},\dots,t_{n,n-1},t_{nn})$
ausgedrückt werden.

\subsubsection{Laplace-Operator}
\begin{tabular}{ll}
Kartesisch: $\Delta u(x,y,z)=\frac{\partial^2u}{\partial x^2}+\frac{\partial^2u}{\partial y^2}+\frac{\partial^2u}{\partial z^2}$
& Zylinder: $\Delta f ( \rho , \phi , z ) = \frac{1}{\rho} \frac{\partial}{\partial \rho}
\left( \rho\,\frac{\partial f}{\partial \rho} \right) +
\frac{1}{\rho^2}\frac{\partial^2 f}{\partial \phi^2} +
\frac{\partial^2 f}{\partial z^2}$ \\
Polar: $\Delta f(r, \phi ) =
\frac{\partial^2 f}{\partial r^2} +
\frac{1}{r}\frac{\partial f}{\partial r} +
\frac{1}{r^2}\frac{\partial^2 f}{\partial \phi^2}$
& Kugel: $\Delta f ( r , \vartheta , \phi ) = \frac{1}{r^2} 
\frac{\partial}{\partial r} \left( r^2  \,\frac{\partial f}{\partial r} \right) +
\frac{1}{r^2 \sin \vartheta}  \frac{\partial}{\partial \vartheta} \left(\sin\vartheta \, \frac{\partial f}{\partial \vartheta} \right) +
\frac{1}{r^2 \sin^2\vartheta}  \frac{\partial^2 f}{\partial \phi^2}$
\end{tabular}

\subsubsection{Umwandlung in System niedriger Ordnung}

\begin{tabular}{ll}
Gegeben:& $F\biggl(x,y,u,\frac{\partial u}{\partial x},\frac{\partial u}{\partial y},
\frac{\partial^2 u}{\partial x^2},\frac{\partial^2 u}{\partial x\partial y},
\frac{\partial^2u}{\partial y^2}\biggr)=0.$\\[0.2cm]
Substitution: & $p=\frac{\partial u}{\partial x},\qquad q=\frac{\partial u}{\partial y}$\\[0.2cm]
Für zweite Ableitungen: & $\frac{\partial^2 u}{\partial x^2}=\frac{\partial p}{\partial x},\quad \frac{\partial^2 u}{\partial x\partial y}=\frac{\partial p}{\partial y}=\frac{\partial q}{\partial x},\quad\frac{\partial^2 u}{\partial y^2}=\frac{\partial q}{\partial y}$\\[0.2cm]
Gleichungssystem 1.Ordnung& $p=\frac{\partial u}{\partial x},\quad q=\frac{\partial u}{\partial y},\quad\frac{\partial p}{\partial y}=\frac{\partial q}{\partial x}$
\end{tabular}

\subsubsection{Notationen einer PDGL, Gebiet $\Omega$}
\begin{minipage}{4cm}
	\begin{tabular}{ll}
	\includegraphics[width=3cm]{Content/Theory/Gebiet}&
	\end{tabular}
\end{minipage}
\begin{minipage}{4cm}	
	\begin{tabular}{ll}
		$\overset{\circ}{\Omega}$ & Innere Punkte\\
		$\partial\Omega$ & Rand\\
		$\overset{\_}{\Omega}$ & Gebiet $\Omega$ und Rand $\partial\Omega$\\
	\end{tabular}
\end{minipage}

Das Gebiet einer PDGL \textbf{muss} offen sein, nur dann ist die partielle
Ableitung überall definiert. Das Gebiet ist offen, wenn um jeden Punkt im Gebiet
$\Omega$ ein kleiner Ball gezeichnet werden kann, welches sich auch im Gebiet $\Omega$ befindet.\\

\begin{minipage}{4cm}
	Kein Gebiet:\\
	\includegraphics[width=4cm]{Content/Theory/gebiet_1.pdf}\\

\end{minipage}
\begin{minipage}{4cm}
  	Gebiet:\\
  	\includegraphics[width=4cm]{Content/Theory/gebiet_2.pdf}\\
\end{minipage}

\textbf{Lösung einer PDGL:}\\
\begin{tabular}{ll}
Gegeben:& Gebiet $\Omega$, PDGL,Randwerte $\partial\Omega$\\
Lösung:& Funktion $u$: $\overset{\_}{\Omega}\rightarrow \mathbb{R}$, PDGL in $\Omega$ und Randwerte auf $\partial\Omega$\\
\end{tabular}

\subsubsection{Klassifikation einer PDGL}
\begin{tabular}{lll}
Ordnung:& \multicolumn{2}{l}{Höchste vorkommende partielle Ableitung}\\
Typ:& Linear: & Linear bei $u, x_1,...,x_n, \frac{\partial u}{\partial x_1},\ldots,\frac{\partial u}{\partial x_n}$\\
& Quasilinear: &  Linear bei $\frac{\partial u}{\partial x_1},\ldots,\frac{\partial u}{\partial x_n}$\\
& Nichtlineare: & Alles andere
\end{tabular}

\subsection{Charakteristiken}
\textbf{Wichtig:} Als Anfangsbedingungen d�rfen \textbf{keine} Charakteristiken verwendet werden, sonst ist die Charakteristik die L�sung (anstatt Fl�che ergibt sich eine Kurve).\\
\textbf{Wichtig:} Die Charakteristik darf den Rand nur einmal durchlaufen.\\
N�tzlich f�r Qasilineare PDGL 1.Ordnung.\todo{Noch mehr?}\\
$$a(x,y,u)\cdot\partFrac{u}{x}+b(x,y,u)\cdot\partFrac{u}{y}=c(x,y,u) \qquad\Rightarrow\qquad a(x,y,u)\cdot\partFrac{u}{x}+b(x,y,u)\cdot\partFrac{u}{y}-c(x,y,u)=0$$\\


\begin{tabular}{ll}
Gebiet:& $\Omega{\ldots|x>0,alle y}$\qquad Randbedingung: $u(0,y_0)=g(y_0)$\\
Vektorielle schreibweise:& $\begin{bmatrix}
	a(x,y,u)\\ b(x,y,u)\\ c(x,y,u)
\end{bmatrix}\cdot 
\underset{\overrightarrow{n}~:~Normale ~auf~ Flaeche}{\underbrace{\begin{bmatrix}
\partFrac{u}{x} & \partFrac{u}{y} & -1
\end{bmatrix}}}=0$\\[1cm]
Tangenten:& $\overrightarrow{t}_x=\begin{bmatrix}1\\0\\ \partFrac{u}{x}\end{bmatrix}\qquad 
			\overrightarrow{t}_y=\begin{bmatrix}0\\1\\ \partFrac{u}{y}\end{bmatrix}$\\[1cm]
L�sungsweg& F�r jeden Anfangspunkt $\begin{bmatrix} 0\\y_0\\g(y_0)\end{bmatrix}$ finde eine Charakteristik, diese nach $x$, $y$ aufl�sen.
\end{tabular}

Beispiel:\\
\begin{enumerate}
	\item PDGL mit Randbedingungen und Definitionsbereich: $\partFrac ux+2\partFrac uy=3$, \qquad $u(0,y_0)=g(y_0)=\sin(y_0)$\\
	Therme in Matrixschreibweise: $\begin{bmatrix}a\\b\\c\end{bmatrix}=\begin{bmatrix}1\\2\\3\end{bmatrix}$
	\item Charakteristiken ausrechnen PDGL $\rightarrow$ DGL: 	$\frac {d}{dt}\begin{bmatrix}x(t)\\y(t)\\u(t)\end{bmatrix}=\begin{bmatrix}1\\2\\3\end{bmatrix}$
	\item DGL's l�sen: 
	$\begin{bmatrix}x\\y\\u\end{bmatrix}=\begin{bmatrix}t+x_0\\t+y_0\\t+u_0\end{bmatrix}$
	\item Anfangsbedingungen einsetzen: $\begin{bmatrix}x\\y\\u\end{bmatrix}=\begin{bmatrix}t+x_0\\t+y_0\\t+u_0\end{bmatrix}\Bigg|_{t=0}=
	\begin{bmatrix}x_0\\y_0\\u_0\end{bmatrix}=\begin{bmatrix}0\\y_0\\\sin(y_0)\end{bmatrix}$\\
	L�sung der DGL ist: $\begin{bmatrix}x\\y\\u\end{bmatrix}=\begin{bmatrix}1\\2\\3\end{bmatrix}\cdot t+ \begin{bmatrix}0\\y_0\\\sin(y_0)\end{bmatrix}$\\
	
	\item Eliminieren aller Variablen ausser $u,x,y$: $u=3x+sin(y-2x)$
	\item Kontrolle
	Resultat ($u=3x+sin(y-2x)$) ableiten und in Aufgabenstellung einsetzen $\partFrac ux+2\partFrac uy=3$ und schauen ob es erf�llt.
	
\end{enumerate}



\subsection{Separation}
Wahl eines geeigneten Koordinatensystems ist wichtig.

\textbf{Beispiels-PDGL: }$\frac1x\partFrac{u}{x}+\frac1y\partFrac{u}{y}=\frac{1}{y^2}$
\begin{enumerate}
\item \textbf{Ansatz: }
	\begin{itemize}
		\item PDGL 1.Ordnung: $U(x,y)=X(x) + Y(y)$
		\item PDGL 2.Ordnung: $U(x,y)=X(x) \cdot Y(y)$ 
	\end{itemize}
	Beispiel: PDGL 1.Ordnung: $U(x,y)=X(x) + Y(y)$ 
\item \textbf{Einsetzen: } Ansatz in PDGL einsetzen.\\[0.4cm]
	Beispiel: $\partFrac{u}{x}=X'(x)$\qquad $\partFrac{u}{y}=Y'(y)$ \quad $\Rightarrow$ \quad $\frac{X'(x)}{x}+\frac{Y'(y)}{y}=\frac{1}{y^2}$
\item \textbf{Separation: } Auf jeder Seite der PDGL darf nur noch eine Variable vorkommen. Die beiden jetzt gew�hnlichen DGL sind �ber eine Konstante gekoppelt (fixieren der Variable). Wahl der Konstante: Wenn Schwingung erwartet wird: $-k^2$, sonst $k$, ausser man weiss es besser ;-).\\[0.4cm]
	Beispiel: $\frac{X'(x)}{x}=k=\frac{1}{y^2}-\frac{Y'(y)}{y}$
\item \textbf{L�sen der DGL's: } Man erh�lt eine Familie von L�sungen\\
	Beispiell:	 
		$X'(x)=k\cdot x \quad\Rightarrow\quad X(x)=\frac12 kX^2+C_x$ \qquad\qquad $Y'(y)=\frac1y-ky \quad\Rightarrow\quad Y(y)=\ln(y)-\frac{ky^2}{2}+C_y$\\
\item \textbf{Gesamtl�sung zusammenbasteln (Linearkombination der L�sungsfamilie), damit Randbedingungen stimmen: }
	Beispiel: $U(x,y)=\frac12 kx^2\frac12 ky^2+ln(y)+C$
\end{enumerate}

\subsection{Transformationen}
\todo{Beschreibung}


\begin{itemize}
\item Der �bergang von Funktionen zu Fourierreihen verwandelt eine partielle
Differentialgleichung in eine Familie gew�hnlicher Differentialgleichungen f�r
die einzelnen Fourier-Koeffizienten.
\item Integraltransformationen k�nnen ein partielle Differentialgleichung in eine
Familie partieller Differentialgleichungen mit weniger Variablen oder sogar
gew�hnlicher Differentialgleichungen verwandeln.
\item Integraltransformationen und die R�cktransformationen k�nnen Formeln
f�r die L�sungen gewisser partieller Differentialgleichungen liefern, und
damit die Frage beantworten, f�r welche Randwertvorgaben die Gleichungen
gut gestellt sind.
\end{itemize}

\subsubsection{Fourierreihe}

$\boxed{u(t,x)=\frac{a_0(t)}{2}+\sum\limits_{k=1}^{\infty}{a_k(t)\cos(kx)+b_k\sin(kx)}}$\\[0.4cm]

\textbf{Beispiel:} Schwingende Saite: $\boxed{\partial_t^2u=\partial_x^2u}$\\

\begin{itemize}
\item Ansatz der Fourieranalyse in PDGL einsetzen:
$\partial_t^2(t,x)=\frac{a_0''(t)}{2}+\sum\limits_{k=1}^{\infty}{a_k''(t)\cos(kx)+b_k''(t)\sin(kx)}$\\
$\partial_x^2(t,x)=-\sum\limits_{k=1}^{\infty}{a_k(t)k^2\cos(kx)+b_k(t)k^2\sin(kx)}$\\
$\partial_t^2(t,x)=\frac{a_0''(t)}{2}+\sum\limits_{k=1}^{\infty}{a_k''(t)\cos(kx)+b_k''(t)\sin(kx)}=-\sum\limits_{k=1}^{\infty}{a_k(t)k^2\cos(kx)+b_k(t)k^2\sin(kx)}=\partial_x^2(t,x)$\\[0.2cm]
$\boxed{\Rightarrow\quad \frac{a_0''(t)}{2}+\sum\limits_{k=1}^{\infty}{\big(a_k''(t)+a_k(t)k^2\big)\cos(kx)+\big(b_k''(t)+b_k(t)k^2\big)\sin(kx)}=0}$
\item Diese Gleichung ist nur L�sbar wenn alle Koeffizienten verschwinden (Fourier-Theorie):\\[0.2cm]
$a_0''(t)=0 \qquad a_k''(t)=-k^2a_k(t)\qquad b_k''(t)=-k^2b_k(t)$
\item Durch die Fouriertransformation wurde die PDGL in ein DGL-System �berf�hrt, die L�sungen sind wohlbekannt:\\[0.2cm]
$a_0(t)=m_0(t)+c_0\qquad a_k(t)=A_k^a\cos(kt)+B_k^a\sin(kt)\qquad b_k(t)=A_k^b\cos(kt)+B_k^b\sin(kt)$
\end{itemize}


\subsubsection{Anfangsbedingungen}
Die Differentialgleichungen f�r die Koeffizienten ak(t) und bk(t) k�nnen erst dann
vollst�ndig gel�st werden, wenn Anfangs oder Randbedingungen gegeben sind.\\
\begin{itemize}
\item Anfangsbedingungen f�r Wellengleichung:\\
\quad $u(0,x)=f(x)\qquad \partFrac{u}{t}=g(x)$
\item Die Funktionen f und g k�nnen auch als Fourrierreihe dargestellt werden:\\
$f(x)=\frac{a_0^f}{2}+\sum\limits_{k=1}^{\infty}{a^f_k\cos(kx)+b^f_k\sin(kx)}$\\[0.2cm]
$g(x)=\frac{a_0^g}{2}+\sum\limits_{k=1}^{\infty}{a^g_k\cos(kx)+b^g_k\sin(kx)}$
\item Zusammen mit dem Ansatz f�r $u(t,x)$ ergeben sich die Gleichungen (f�r $t=0$):\\
$\frac{a_0(0)}{2}+\sum\limits_{k=1}^{\infty}{a_k(0)\cos(kx)+b_k(0)\sin(kx)}=\frac{a_0^f}{2}+\sum\limits_{k=1}^{\infty}{a_k^f\cos(kx)+b^f_k\sin(kx)}$\\[0.2cm]
$\frac{a'_0(0)}{2}+\sum\limits_{k=1}^{\infty}{a_k'(0)\cos(kx)+b_k'(0)\sin(kx)}=\frac{a_0^g}{2}+\sum\limits_{k=1}^{\infty}{a_k^g\cos(kx)+b^g_k\sin(kx)}$
\item Koeffizientenvergleich ergibt:\\
$a_k(0)=a_k^f\qquad a_k'(0)=a_k^g\qquad b_k(0)=b_k^f\qquad b_k'(0)=b_k^g$
\item Die vollst�ndige L�sung ist damit:\\
$u(t,x)=\frac{a_0^g(t)+a_0^f}2+\sum\limits_{k=1}^{\infty}{\left(a_k^f\cos(kt)+\frac 1k a_k^2\sin(kt)\right)\cos(kx)+\left(b_k^f\cos(kt)+\frac 1k b_k^2\sin(kt)\right)}\sin(kx)$
\end{itemize}

\subsubsection{Inhomogene Wellengleichung}

Das Verfahren l�sst sich auch auf die inhomogene Wellengleichung verallgemeinern. Das St�rglied wird dabei ebenfalls als Fourierreihe entwickelt.

$\partial_t^2u-\partial_x^2u=f \qquad \Rightarrow \qquad f(t,x)=\frac{a_0^f(t)}{2}+\sum\limits_{k=1}^{\infty}{a^f_k(t)\cos(kx)+b^f_k\sin(kx)}$












\subsection{PDGL 2.Ordnung}
Lineare partielle Differentialgleichungen zweiter Ordnung haben die Form:
$\boxed{\sum\limits_{i,j=1}^{n}{a_{ij}\partial_i\partial_j u}+\sum\limits_{i=1}^{n}{b_i\partial_i u}+cu=f}$

\subsubsection{Klassifikation}
Eigenwertberechnung: 
\begin{enumerate}
  \item Symmetrische Matrix aufstellen und $\lambda$ in der Diagonalen abziehen. Z.B.: $A = \begin{pmatrix}
    \partial_x^2 & \partial_x \partial_y \\
    \partial_y \partial_x  & \partial_y^2
  \end{pmatrix}$\\
  Bei diagonalen Matrizen entsprechen die Eigenwerte den Diagonaleinträgen.
  \item Determinante gleich 0 setzen: $\det(\mathbf{A}-\lambda \mathbf{I}) = 0\quad\Rightarrow\quad \lambda_i$
  \item Gleichung lösen
\end{enumerate}
\begin{tabular}{|l||l|l|l|l|}
\hline
\multirow{2}{*}{Klasse}&\multicolumn{3}{|c|}{Anzahl Eigenwerte}&\multirow{2}{*}{Beispiel}\\
&Positiv&Negativ&Verschwindend&\\
\hline
hyperbolisch& n-1 & 1 & 0 & Wellengleichung\\
\hline
parabolisch& n-1 & 0 & 1 & Wärmeleitung\\
\hline
elliptisch&	n & 0 & 0 & Potential\\
\hline
ultrahyperbolisch & >1 & >1 & 0 & -\\
\hline
\end{tabular}

\subsection{Elliptische PDGL}
\todo{Beschreibung}
%\input{Content/Theory/Parabolische.tex}
\subsection{Hyperbolische PDGL}
\begin{minipage}{14cm}
	PDGL: $\partFrac{^2u}{t^2}-a^2\partFrac{^2 u}{x^2}=0\qquad \Omega=\left\{(x,t)|t>0\right\}\qquad u_0=u(x_0,0)$\\
	
	Trick: $(\partial_t +a\partial_x)(\partial_t-a\partial_x)u=(\partial_t^2-a^2\partial_x^2)u=0$\qquad(für konstante Geschwindigkeit $a$)\\
	
	Zwei mögliche Lösungen: $\underset{\text{\cfbox{red}{Nach rechts laufende Welle}}}{\underbrace{(\partial_t +a\partial_x)u=0}}\qquad \underset{\text{\cfbox{black}{Nach links laufende Welle}}}{\underbrace{(\partial_t-a\partial_x)u=0}}$\\
	
	Lösung mittels Charakteristiken: $\partFrac{}{s}
	\begin{Bmatrix}
		x(s)\\
		t(s)\\
		u(s)
	\end{Bmatrix}=
	\begin{Bmatrix}
		\pm a\\
		1\\
		0
	\end{Bmatrix}
	\begin{array}{ll}
		\Rightarrow&x=\pm as +x_0\\
		\Rightarrow&t= s +t_0=s\qquad (t_0=0)\\
		\Rightarrow&u=u_0\\
	\end{array}
	$\\
	
	$x=\pm at+x_0\quad\Rightarrow\quad x_0=x\mp at\quad\Rightarrow\quad u(x,t)=u_0(x\mp at)$\\
	
	Allgemeine Lösung aus Linearkombination: $\boxed{u(x,t)=u_+(x+at)+u_-(x-at)}$\\
	
	$\Rightarrow$ Es werden \textbf{zwei} Anfangsbedingungen benötigt um $u_+$ \textbf{und} $u_-$ zu bestimmen.\\
	
	z.B.: $u(x,0)=u_0(x)\qquad \partFrac{u}{t}(x,0)=g_0(x)$
	\end{minipage}
	\hfill
	\begin{minipage}{5cm}
	\includegraphics[width=5cm]{Content/Theory/linksRechts}
	\end{minipage}


