\subsection{PDGL 2.Ordnung}
Lineare partielle Differentialgleichungen zweiter Ordnung haben die Form:
$\boxed{\sum\limits_{i,j=1}^{n}{a_{ij}\partial_i\partial_j u}+\sum\limits_{i=1}^{n}{b_i\partial_i u}+cu=f}$

\subsubsection{Klassifikation}
Eigenwertberechnung: 
\begin{enumerate}
  \item Symmetrische Matrix aufstellen und $\lambda$ in der Diagonalen abziehen. Z.B.: $A = \begin{pmatrix}
    \partial_x^2 & \partial_x \partial_y \\
    \partial_y \partial_x  & \partial_y^2
  \end{pmatrix}$\\
  Bei diagonalen Matrizen entsprechen die Eigenwerte den Diagonaleinträgen.
  \item Determinante gleich 0 setzen: $\det(\mathbf{A}-\lambda \mathbf{I}) = 0\quad\Rightarrow\quad \lambda_i$
  \item Gleichung lösen
\end{enumerate}
\begin{tabular}{|l||l|l|l|l|}
\hline
\multirow{2}{*}{Klasse}&\multicolumn{3}{|c|}{Anzahl Eigenwerte}&\multirow{2}{*}{Beispiel}\\
&Positiv&Negativ&Verschwindend&\\
\hline
hyperbolisch& n-1 & 1 & 0 & Wellengleichung\\
\hline
parabolisch& n-1 & 0 & 1 & Wärmeleitung\\
\hline
elliptisch&	n & 0 & 0 & Potential\\
\hline
ultrahyperbolisch & >1 & >1 & 0 & -\\
\hline
\end{tabular}