\subsection{Transformationen}
\todo{Beschreibung}


\begin{itemize}
\item Der �bergang von Funktionen zu Fourierreihen verwandelt eine partielle
Differentialgleichung in eine Familie gew�hnlicher Differentialgleichungen f�r
die einzelnen Fourier-Koeffizienten.
\item Integraltransformationen k�nnen ein partielle Differentialgleichung in eine
Familie partieller Differentialgleichungen mit weniger Variablen oder sogar
gew�hnlicher Differentialgleichungen verwandeln.
\item Integraltransformationen und die R�cktransformationen k�nnen Formeln
f�r die L�sungen gewisser partieller Differentialgleichungen liefern, und
damit die Frage beantworten, f�r welche Randwertvorgaben die Gleichungen
gut gestellt sind.
\end{itemize}

\subsubsection{Fourierreihe}

$\boxed{u(t,x)=\frac{a_0(t)}{2}+\sum\limits_{k=1}^{\infty}{a_k(t)\cos(kx)+b_k\sin(kx)}}$\\[0.4cm]

\textbf{Beispiel:} Schwingende Saite: $\boxed{\partial_t^2u=\partial_x^2u}$\\

\begin{itemize}
\item Ansatz der Fourieranalyse in PDGL einsetzen:
$\partial_t^2(t,x)=\frac{a_0''(t)}{2}+\sum\limits_{k=1}^{\infty}{a_k''(t)\cos(kx)+b_k''(t)\sin(kx)}$\\
$\partial_x^2(t,x)=-\sum\limits_{k=1}^{\infty}{a_k(t)k^2\cos(kx)+b_k(t)k^2\sin(kx)}$\\
$\partial_t^2(t,x)=\frac{a_0''(t)}{2}+\sum\limits_{k=1}^{\infty}{a_k''(t)\cos(kx)+b_k''(t)\sin(kx)}=-\sum\limits_{k=1}^{\infty}{a_k(t)k^2\cos(kx)+b_k(t)k^2\sin(kx)}=\partial_x^2(t,x)$\\[0.2cm]
$\boxed{\Rightarrow\quad \frac{a_0''(t)}{2}+\sum\limits_{k=1}^{\infty}{\big(a_k''(t)+a_k(t)k^2\big)\cos(kx)+\big(b_k''(t)+b_k(t)k^2\big)\sin(kx)}=0}$
\item Diese Gleichung ist nur L�sbar wenn alle Koeffizienten verschwinden (Fourier-Theorie):\\[0.2cm]
$a_0''(t)=0 \qquad a_k''(t)=-k^2a_k(t)\qquad b_k''(t)=-k^2b_k(t)$
\item Durch die Fouriertransformation wurde die PDGL in ein DGL-System �berf�hrt, die L�sungen sind wohlbekannt:\\[0.2cm]
$a_0(t)=m_0(t)+c_0\qquad a_k(t)=A_k^a\cos(kt)+B_k^a\sin(kt)\qquad b_k(t)=A_k^b\cos(kt)+B_k^b\sin(kt)$
\end{itemize}


\subsubsection{Anfangsbedingungen}
Die Differentialgleichungen f�r die Koeffizienten ak(t) und bk(t) k�nnen erst dann
vollst�ndig gel�st werden, wenn Anfangs oder Randbedingungen gegeben sind.\\
\begin{itemize}
\item Anfangsbedingungen f�r Wellengleichung:\\
\quad $u(0,x)=f(x)\qquad \partFrac{u}{t}=g(x)$
\item Die Funktionen f und g k�nnen auch als Fourrierreihe dargestellt werden:\\
$f(x)=\frac{a_0^f}{2}+\sum\limits_{k=1}^{\infty}{a^f_k\cos(kx)+b^f_k\sin(kx)}$\\[0.2cm]
$g(x)=\frac{a_0^g}{2}+\sum\limits_{k=1}^{\infty}{a^g_k\cos(kx)+b^g_k\sin(kx)}$
\item Zusammen mit dem Ansatz f�r $u(t,x)$ ergeben sich die Gleichungen (f�r $t=0$):\\
$\frac{a_0(0)}{2}+\sum\limits_{k=1}^{\infty}{a_k(0)\cos(kx)+b_k(0)\sin(kx)}=\frac{a_0^f}{2}+\sum\limits_{k=1}^{\infty}{a_k^f\cos(kx)+b^f_k\sin(kx)}$\\[0.2cm]
$\frac{a'_0(0)}{2}+\sum\limits_{k=1}^{\infty}{a_k'(0)\cos(kx)+b_k'(0)\sin(kx)}=\frac{a_0^g}{2}+\sum\limits_{k=1}^{\infty}{a_k^g\cos(kx)+b^g_k\sin(kx)}$
\item Koeffizientenvergleich ergibt:\\
$a_k(0)=a_k^f\qquad a_k'(0)=a_k^g\qquad b_k(0)=b_k^f\qquad b_k'(0)=b_k^g$
\item Die vollst�ndige L�sung ist damit:\\
$u(t,x)=\frac{a_0^g(t)+a_0^f}2+\sum\limits_{k=1}^{\infty}{\left(a_k^f\cos(kt)+\frac 1k a_k^2\sin(kt)\right)\cos(kx)+\left(b_k^f\cos(kt)+\frac 1k b_k^2\sin(kt)\right)}\sin(kx)$
\end{itemize}

\subsubsection{Inhomogene Wellengleichung}

Das Verfahren l�sst sich auch auf die inhomogene Wellengleichung verallgemeinern. Das St�rglied wird dabei ebenfalls als Fourierreihe entwickelt.

$\partial_t^2u-\partial_x^2u=f \qquad \Rightarrow \qquad f(t,x)=\frac{a_0^f(t)}{2}+\sum\limits_{k=1}^{\infty}{a^f_k(t)\cos(kx)+b^f_k\sin(kx)}$











