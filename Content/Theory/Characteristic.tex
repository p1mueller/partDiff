\subsection{Charakteristiken}
\textbf{Wichtig:} Als Anfangsbedingungen d�rfen \textbf{keine} Charakteristiken verwendet werden, sonst ist die Charakteristik die L�sung (anstatt Fl�che ergibt sich eine Kurve).\\
\textbf{Wichtig:} Die Charakteristik darf den Rand nur einmal durchlaufen.\\
N�tzlich f�r Qasilineare PDGL 1.Ordnung.\todo{Noch mehr?}\\
$$a(x,y,u)\cdot\partFrac{u}{x}+b(x,y,u)\cdot\partFrac{u}{y}=c(x,y,u) \qquad\Rightarrow\qquad a(x,y,u)\cdot\partFrac{u}{x}+b(x,y,u)\cdot\partFrac{u}{y}-c(x,y,u)=0$$\\


\begin{tabular}{ll}
Gebiet:& $\Omega{\ldots|x>0,alle y}$\qquad Randbedingung: $u(0,y_0)=g(y_0)$\\
Vektorielle schreibweise:& $\begin{bmatrix}
	a(x,y,u)\\ b(x,y,u)\\ c(x,y,u)
\end{bmatrix}\cdot 
\underset{\overrightarrow{n}~:~Normale ~auf~ Flaeche}{\underbrace{\begin{bmatrix}
\partFrac{u}{x} & \partFrac{u}{y} & -1
\end{bmatrix}}}=0$\\[1cm]
Tangenten:& $\overrightarrow{t}_x=\begin{bmatrix}1\\0\\ \partFrac{u}{x}\end{bmatrix}\qquad 
			\overrightarrow{t}_y=\begin{bmatrix}0\\1\\ \partFrac{u}{y}\end{bmatrix}$\\[1cm]
L�sungsweg& F�r jeden Anfangspunkt $\begin{bmatrix} 0\\y_0\\g(y_0)\end{bmatrix}$ finde eine Charakteristik, diese nach $x$, $y$ aufl�sen.
\end{tabular}

Beispiel:\\
\begin{enumerate}
	\item PDGL mit Randbedingungen und Definitionsbereich: $\partFrac ux+2\partFrac uy=3$, \qquad $u(0,y_0)=g(y_0)=\sin(y_0)$\\
	Therme in Matrixschreibweise: $\begin{bmatrix}a\\b\\c\end{bmatrix}=\begin{bmatrix}1\\2\\3\end{bmatrix}$
	\item Charakteristiken ausrechnen PDGL $\rightarrow$ DGL: 	$\frac {d}{dt}\begin{bmatrix}x(t)\\y(t)\\u(t)\end{bmatrix}=\begin{bmatrix}1\\2\\3\end{bmatrix}$
	\item DGL's l�sen: 
	$\begin{bmatrix}x\\y\\u\end{bmatrix}=\begin{bmatrix}t+x_0\\t+y_0\\t+u_0\end{bmatrix}$
	\item Anfangsbedingungen einsetzen: $\begin{bmatrix}x\\y\\u\end{bmatrix}=\begin{bmatrix}t+x_0\\t+y_0\\t+u_0\end{bmatrix}\Bigg|_{t=0}=
	\begin{bmatrix}x_0\\y_0\\u_0\end{bmatrix}=\begin{bmatrix}0\\y_0\\\sin(y_0)\end{bmatrix}$\\
	L�sung der DGL ist: $\begin{bmatrix}x\\y\\u\end{bmatrix}=\begin{bmatrix}1\\2\\3\end{bmatrix}\cdot t+ \begin{bmatrix}0\\y_0\\\sin(y_0)\end{bmatrix}$\\
	
	\item Eliminieren aller Variablen ausser $u,x,y$: $u=3x+sin(y-2x)$
	\item Kontrolle
	Resultat ($u=3x+sin(y-2x)$) ableiten und in Aufgabenstellung einsetzen $\partFrac ux+2\partFrac uy=3$ und schauen ob es erf�llt.
	
\end{enumerate}


